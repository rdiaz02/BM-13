\documentclass[a4paper,11pt,twoside]{article}
\usepackage[absolute,overlay]{textpos}
\usepackage[english]{babel} %% needed for refs not to get a "~"

\usepackage{lscape}
\usepackage{array}
\usepackage{geometry}
\geometry{verbose,a4paper,tmargin=23mm,bmargin=26mm,lmargin=30mm,rmargin=30mm}

\usepackage{setspace}
\singlespacing

\usepackage{graphicx}

\usepackage[latin1]{inputenc}
\usepackage{times}
\usepackage[T1]{fontenc}

\setcounter{secnumdepth}{5} %% paragraphs in toc
\setcounter{tocdepth}{5}


\usepackage{datetime} %% but I think I am not using it


\usepackage{tikz}
\usetikzlibrary{arrows,shapes,positioning}

\tikzset{
  % Define standard arrow tip
  >=stealth',
  % Define style for boxes
  punkt/.style={
    rectangle,
    rounded corners,
    draw=black, %very thick,
    %maximum width=6em,
    %text width=6.5em,
    minimum height=2em,
    text centered},
  punkt2/.style={
    rectangle,
    rounded corners,
    draw=black, %very thick,
    text width=2.35cm,
    %text width=6.5em,
    minimum height=2em,
    inner sep = 2pt,
    text centered},
  punkt3/.style={
    rectangle,
%    rounded corners,
    draw=black, %very thick,
%    text width=1.cm,
    %text width=6.5em,
    minimum height=2em,
    inner sep = 3pt,
    text centered},
  punkk/.style={
    rectangle,
    rounded corners=3mm,
    draw=black, %very thick,
    %text width=6.5em,
    minimum height=2em,
    text centered},
  ell1/.style={
    %ellipse,
    %draw = black, very thick,
    text width=1.7cm,
    inner sep = 0pt,
    text centered},
  % Define arrow style
  pil/.style={
    ->,
    thick,
    shorten <=2pt,
    shorten >=2pt,}
}




\usepackage{verbatim}
\usepackage{amsmath}
\usepackage[hyphens]{url}
\usepackage[pdftex]{hyperref}
%\usepackage{cite}

\usepackage{url}
\usepackage{xcolor}
\definecolor{light-gray}{gray}{0.72}
\newcommand{\cyan}[1]{{\textcolor {cyan} {#1}}}
\newcommand{\blu}[1]{{\textcolor {blue} {#1}}}
\newcommand{\Burl}[1]{\blu{\url{#1}}}
\newcommand{\Curl}[1]{{\small \url{#1}}} %% you need to scape #

\newcommand{\lgray}[1]{{\textcolor {light-gray} {#1}}}
\newcommand{\red}[1]{{\textcolor {red} {#1}}}
\newcommand{\green}[1]{{\textcolor {green} {#1}}}
\newcommand{\mg}[1]{{\textcolor {magenta} {#1}}}
\newcommand{\og}[1]{{\textcolor {PineGreen} {#1}}}
%\newcommand{\code}[1]{\texttt{\slshape\footnotesize #1}}
%\newcommand{\code}[1]{\texttt{\slshape #1}}
\newcommand{\code}[1]{\texttt{#1}}
\newcommand{\opage}[1]{\texttt{[p.\ #1]}}
\newcommand{\myverb}[1]{{\footnotesize\texttt {\textbf{#1}}}}
\newcommand{\Rnl}{\ +\qquad\ }
\newcommand{\Emph}[1]{\emph{\mg{#1}}}

\providecommand{\tabularnewline}{\\}

%\usepackage{harvard}
\usepackage[authoryear, round, sort]{natbib}
\bibliographystyle{myagsm} %%bbs and chicago does not show URLs 
%%bbs and chicago does not show URLs 

% /texmf/bibtex/bst/myagsm.bst
% and then do texhash

%% Algorithm2e needs to come after natbib!!
\usepackage[boxed,vlined,linesnumbered]{algorithm2e}


%% \usepackage{enumitem} %% not clear I need it
\usepackage{enumerate}

%% an ugly hack, because I cannot get anything else to work

% \newlist{exsection}{enumerate}{10}
% \setlist[exsection]{label*=\thesection.\arabic*.}

% \newlist{exsubsection}{enumerate}{10}
% \setlist[exsubsection]{label*=\thesubsection.\arabic*.}

% \newlist{exsubsubsection}{enumerate}{10}
% \setlist[exsubsubsection]{label*=\thesubsubsection.\arabic*.}


% \DeclareMathOperator*{\argmax}{arg\,max}
% \underset{x}{\operatorname{argmax}}



\title{Statistics for omics. \\ \large Class notes for BM-13, ``Bioinform�tica Avanzada y
Biolog�a de Sistemas'', 2014-2015.}

\author{Ram�n Diaz-Uriarte\\
              Dept. Bioqu�mica\\Universidad Aut�noma de Madrid \\ 
              \texttt{ramon.diaz@iib.uam.es} \\ 
              \Burl{http://ligarto.org/rdiaz} }

%\date{}
% \date{\the\year-\the\month-\the\day}

\date{\gitAuthorDate\ {\footnotesize (Release\gitRels: Rev:
    \gitAbbrevHash)}}


\makeindex


\begin{document}

\maketitle
\vspace{-0.5cm}


\tableofcontents
\clearpage





\section{Pedagogical and scientific objectives of this module}


The main objective of this module is to make you familiar with the ``big
themes'' when using statistics for the analysis of omics data. In
particular, at the end of this module you should:

\begin{itemize}
\item  Be able to identify why multiple testing is such a serious and
  pervasive problem.

\item Have a clear understanding of the need for decent estimates of
  classification error and some of the more common procedures to obtain
  them.

\item Be familiar with survival analysis.

\item Be aware of recent trends in classification/prediction
  (penalization, wisdom of crowds, etc).

\item Understand what clustering is and is not.

\item Be a critical user of applications for omics data: ask what you are
  being given, and understand whether what the application does is what
  you \textbf{really} need to have done.

\item Understand some of the questions a statistician might ask you when
  you go looking for help.

\item Be able to read the primary bioinformatics literature to understand
  what a method does.

\item Be able to read the primary bioinformatics literature with reviews
  about methods/approaches.
\end{itemize}


There are no notes for this module as such, only a guide to reading and
asking questions about a few papers.


% Most of the above objectives will be covered if you can understand all of
% \cite{DupuySimon07}.

\section{Readings}


We will change focus here with respect to previous blocks. Here, we will
do a fair amount of reading. That means \textbf{you} need to do a fair
amount of reading. We will not focus closely on the math details of
algorithms, but we will need to keep a clear idea of what it is we are
trying to accomplish (e.g., when we say ``I want to build a classifier
from omics data'', what do we \textbf{really} mean?, and why would we want
to do that?).




We will start with a presentation, ``Stats-bioinfo-intro.pdf'', to cover
the main issues and problems. This material should be familiar to some of
you (4� bioqu�mica) and much of the first goes over issues we discussed in
BM-1. 

Then, the key here is for you to read several papers from a range of
journals. We will start with a review of issues, and then we will read
several papers from Bioinformatics/Computational Biology journals.



First, the review: you need to understand ALL of ``Dupuy\_Simon+supp2.pdf''
(they make a couple of contentious claims, but you need to see those too
and be ready to counter argue). Next, read (it is short)
``katz-kattan.pdf''.


Now, go to ``Tarca-IMPROVER.pdf''. Read it in the way you would read it if
you were trying to get an idea of the current state of the art in the
field. I have provided you my annotated copy from my first reading of the
paper; it is an example of how I do these things (just an example, my way,
not a canonical way). I think it took me less than 20 minutes. What took
me much longer, in comparison, was to find a clear explanation of their
metrics (their supplementary material is provided as a Word file, and when
I open it with LibreOffice, the equations are not displayed properly); I
have added to the paper a presentation by them I found that shows things
clearly.





Let us jump now to survival analysis. Read
``Survival-breast-cancer-PLoS-Comp-Biol.pdf'', again with a spirit similar
to the above.  For completeness I am also including another paper they
published\footnote{Is this getting close to ``double publicaton''? Think
  about this}. You do not need to read the second paper: it is there for completeness
and to address possible questions you might have. But you do need to read
the PLoS Computation Biology one.



Finally, read
``Boulesteix-Hothorn-additional-pred-value-covars.pdf''. This is not easy,
and I want you to use \textbf{your} common sense to skip over details and
focus on the key ideas. The details are there if you need them, but in a
first (or second or third) reading you might not need to follow everything
in every detail.


If we have time, we will go back to classification, but now a critical one
about some recent (fancy?)  procedures:
``composite-feature-classification-breast-cancer.pdf''. Read it. What are
they doing and why? Is this of any use for things you might want to do?


%% Fix references in Mendeley and in here!! citeulike ids are a pain!!

% \begin{enumerate}
% \item Start by reading \cite{DupuySimon07}.

% \item You might want to look at the notes about p-values,
%   classification, etc, from BM-1.

% \item Take a look at the PDF of the presentation
%   (Presentation-stats-omics.pdf). Skip most of the survival analysis
%   stuff; get just the general ideas (the rest is there for reference).

% \item If it helps, you can take a look at \cite{Diaz-Uriarte2005}; or you
%   can just skip that material completely.

% \item All of \cite{DupuySimon07} should now be clear. And all, or most, of
%   the presentation.  

% \item Read \cite{KatzKattan05}. % katz and kattan

% \item Take a look at \cite{Shi2010}. Read just the main paper, but beware
%   that there is additional info in the supplementary material.

% % \item Go (quicRead \cite{citeulike:3984020}. Most of the paper should be
% %   understandable (except for maybe a few sections). El de Braga
% %   Neto. not for now.)
% \item Take a look at \cite{Boulesteix2010}. What I expect from you is to
%   understand the problem, the general ideas of the solution developed, and
%   try to follow the logic. This paper is the one that addresses the
%   objective listed as ``Be able to read the primary bioinformatics
%   literature to understand what a method does.''.  
% \end{enumerate}



Please, please, \textbf{I REALLY DO EXPECT YOU TO DO THE READING ABOVE}
before we are done (i.e., not all of it for the first day, but definitely
before the last one). Why? Understanding that reading the primary
Bionfo/computational biology literature is doable is a key part of this
class. And with the reading done, we can hopefully have lively discussions
in class going over the papers.

% \section{Exercises and activities}
% \label{sec:exercises-activities}


\section{Exercises}


A few will be familiar from BM-1. You \textbf{must} answer each of
these. And you want to know how to answer them, since questions similar to
these could be in the exam. \textbf{The most interesting exercise, however, will
be class discussion of the papers}.

%% Answers come from biophysics masters

\begin{enumerate}


\item FDR refers to:

  \begin{enumerate}[(a)]
  \item The expected proportion of false positives in a list of selected genes.
  \item The probability that a given gene is a false positive.
  \item The p-value before multiple testing correction.
  \item The probability of the Type I error rate.
  \end{enumerate}


\item Local FDR (or ``Posterior error probability'') refers to:
  \begin{enumerate}[(a)]
  \item The estimated number of false positives in a set of features.
  \item The probability that a given feature is a false positive.
  \item The adjusted p-value after multiple testing correction.
  \item The probability of the Type I error rate.

  \end{enumerate}




\item A microarray study uses an array with 10000 genes. If there are no
differentially expressed genes, but we consider as "differentially
expressed" any gene with a p-value < 0.01, then on average we will get:

 \begin{enumerate}[(a)]
\item 10 false positives for every true positive
\item 50 false positives
\item 100 false positives
\item 500 false positives
\end{enumerate}


\item When building a classifier from gene expression (with 30000 original
genes) data and a training sample of 100 samples, which of the following
strategies is likely to result in a "better" classifier (where better
means capable of predicting, with low error, the value of new samples):
\begin{enumerate}[(a)]
\item Using 100 genes, 50 of them with small p-values and 50 with large p-values.
\item Using very large FDRs.
\item A simple algorithm (e.g., nearest mean, 3-NN) that uses the values from all 30000 genes.
\item A simple algorithm (e.g., nearest mean, 3-NN) that uses the values from 10 genes with small p-values.
\end{enumerate}


\item A study uses 30 patients that have developed metastasis and 40 patients
that have not developed metastasis. Using an array of 40000 genes, a final
set of 68 genes are selected to differentiate (to classify) between
metastatis and no-metastasis using a Nearest Mean algorithm. Which of the
following is true?
\begin{enumerate}[(a)]
\item With 70 patients and 68 genes, we are making optimal use of the
available information, so we are likely to obtain a good classifier (one
that will perform well with new samples).
\item With 70 patients and 68 genes, we will probably have a serious problem of overfitting.
\item With 70 patients and 68 genes, we should use a 3-Nearest Neighbor algorithm.
\item With 70 patients, we probably want to use quadratic discriminant analysis or random forests.
\end{enumerate}


\item When analyzing time-to-event (or survival data) with microarray data:
\begin{enumerate}[(a)]
\item To avoid errors from censoring and truncation, we should divide the samples in 5 to 10 groups, and use methods for class prediction.
\item We should divide the samples into two groups, and use a binary classifier.
\item We can use a Cox model, and examine the relationship between survival and (some subset of ) the genes.
\item Survival data are never analyzed with gene expression studies.
\end{enumerate}

\item You carry out a t-test two compare the gene expression of gene XYZ between
patients with and without prostate cancer and you obtain a t statistic =
2.4, with a p-value = 0.02. What does the p-value mean?

\begin{enumerate}[(a)]
\item The probability that the mean of gene XYZ is different between  the two groups is 0.02.
\item The probability of the alternative hypothesis is 0.98 = (1 - 0.02).
\item The probability of the null hypothesis (that there is no difference in  the mean expression of XYZ between the two groups) is 0.02.
\item The probability of obtaining a statistic as large as, or  larger than, 1.2 is 0.02, if there is no difference in the mean between  the two groups of patients.
\item The probability of obtaining a statistic as large as, or  larger than, 2.4 is 0.05, if there is no difference in the mean between  the two groups of patients.
\item The probability of obtaining a statistic as large as, or  larger than, 2.4 is 0.02, if there is no difference in the mean between  the two groups of patients.

\end{enumerate}





\item You carry out a microarray study comparing the expression of 20000 genes
between 10 subjects with colon cancer and 150 subjects without. If there
are no genes with any real difference in expression between the two groups
of subjects ...
\begin{enumerate}[(a)]
\item You can expect to find about 200 genes with p-values  less or equal to
   0.01.
 \item You can expect to find about 200 genes with p-values less or equal to 0.1.
 \item You can expect to find about 3200 = (20000 * 0.001 * (150 + 10)) genes
   with p-values below 0.001.
 \item There will be no genes with p-values below 0.05.
 \item Only genes with large fold differences will have small p-values.
\end{enumerate}







\item You just read two papers; the first (A) is searching for differential
expression between bald and non-bald males; the second (B) is searching
for differential expression between patients with and without
diabetes. Study A has examined a total of 100 genes. Study B has examined
a total of 10000 genes. Study A has found 22 genes with p-values < 0.05
and 12 genes with p-values < 0.01. Study B has found 510 genes with
p-values < 0.05 and 98 with p-values < 0.01. 
\begin{enumerate}[(a)]
\item     You are much more interested in study A. Those results provide some
    evidence of association between gene expression and baldness.

  \item      Study B is not really interesting: if there were no differences
     between the two groups of subjects, you could expect to find at least
     as many "significant" genes as this study found.

   \item      Study A is not really interesting: if there were no differences
     between the two groups of subjects, you could expect to find at least
     as many "significant" genes as this study found.

   \item      You are much more interested in study B. Many more genes are showing
     a difference.

   \item     Both studies show the same degree of evidence that there really is an
    effect of the condition on the expression of genes.
\end{enumerate}


\item A study has developed a classifier to predict risk of breast cancer using
gene expression data. This study has used only females, and they report
that they have found a classifier, using only five genes (two of them
related to progesterone receptors), that can predict the risk of breast
cancer with only a 2\% error. Using only the data from that study, can you
estimate the likely error of that classifier if it were applied to males?

\begin{enumerate}[(a)]
\item Yes, if you use cross-validation.
\item    No, you cannot, since there were no males in the study.
\end{enumerate}


\item Suppose I have a gene expression study with 120 subjects, 60 with
 prostate cancer and 60 without and I want to develop a diagnostic tool to
 classify (predict) if a subject has prostate cancer or not based on his
 gene expression. I follow this steps:
\begin{enumerate}[1.]

	\item Before developing the classifier with my data, I leave aside a
          set (call it B) of 20 subjects (10 with and 10 without prostate
          cancer).

	\item So I have two sets, set A with 100 subjects and set B with 20
          subjects.

	\item I develop the classifier using set A and I evaluate the
          prediction error with set B. I report the error of my classifier
          as the error I make when classifying the subjects of set B.
\end{enumerate}

Is this an example of cross-validation?


\item For the detection of breast cancer there are several well
  established clinical indicators. Suppose a paper claims that, using high
  throughput techniques (e.g., microarrays) they can provide improved
  diagnostic accuracy.  What should they do to convince you of the
  clinical utility of their method?




\item Dupuy and Simon, in their paper "Critical Review of Published Microarray Studies for
Cancer Outcome and Guidelines on Statistical Analysis and Reporting" say
in their Checlist number 9: "Don't Transform time-to-outcome data into a binary outcome variable if the goal is to predict groups with different survival probabilities."

Why? What is a better approach?

\item Dupuy and Simon, in their paper "Critical Review of Published
  Microarray Studies for Cancer Outcome and Guidelines on Statistical
  Analysis and Reporting" say in their Checlist number 14: ``Don't Use
  class discovery methods if you are interested in classifying new samples
  in the future." Explain what they mean (remember: class discovery is
  another name for clustering).


\item Suppose you want to develop an algorithm to detect early colorectal cancer.  Why do you need cross-validation,  bootstrap, or similar procedures to assess the classification error?

\item Explain, briefly, the differences between clustering and
  classification. List one scenario where clustering is the right approach
  (and classification won't work) and one scenario where classification is
  the right approach (and clustering is not).


\item What is this wisdom of crowds thing? Where do you see it in the
  papers we've read? 

\item Again: read the papers, annotate them, think through them, and how
  they relate to your own work.
\end{enumerate}



\section{Web-based applications activities}

These are things you might play around if you want to. We will not have
time to go over this.

% \textbf{(If time permits; do not try it the first day!!!)}


% Make sure you do the activities and understand the output. We will discuss
% these in class and they could also be part of the exam.


There might be details you do not understand; in that case, go to the help
of the application. Note that Pomelo II contains tutorials, and a much
more extensive help system.

In the following activities, if some figures seem missing, or if it takes
forever to load, or you get 502 errors, etc, refresh the page. 

% (And I am the [co-]author of these applications, which means I am more
% familiar and frustrated with their shortcomings. So do not be afraid to be
% critical).


\begin{enumerate}

\item Differential expression. We will use Pomelo II
  (\Burl{http://pomelo2.iib.uam.es/}) as an example.
  \begin{enumerate}
  \item Go to
    \Burl{http://pomelo2.iib.uam.es/Examples/LeukemiaGolub/results.html}. All
    of the results should be easy to understand. Make sure you get to the
    bottom of the page. Those dendrograms and heatmaps are very common in
    papers. 

  \item Go to
    \Burl{http://pomelo2.iib.uam.es/Examples/index.html}. Select
    ``Srbct example'' from the box of called ``TRY IT OUT'' (under
    ``ANOVA and other linear models''). Follow the instructions. 

    What is the output shown in the main results page? What is being
    tested or examined there? (That is an F test, and this was a
    four-class problem).


    Go to ``Compare classes'' (at the top of the main results). What can
    you do there? How many two-class comparisons can you make? Does it
    make sense to try all?

    Change the FDR criterion. What happens? 

    Draw Venn diagrams. What do they tell you? What happens when you
    change the FDR? Does it make sense? And how many comparison can you
    show? 



  \item Go to
    \Burl{http://pomelo2.iib.uam.es/Examples/index.html}. Select
    ``Breast example'' from the box of called ``TRY IT OUT'' (under
    ``ANOVA and other linear models'').  Follow the steps (i.e., select
    the Anova linear models test, and select ``Send covariables'). What is
    this all about? What are we doing? Play around with it, and try to see
    what happens; you can run it with some or all or none of the
    covariables. Start, for instance, by using Age.

    Think about a situation where a plot of the covariables would probably
    indicate problems or trouble. What would that plot look like? 

    Think about a situation where you really, really, think you should
    include a covariate: how would results differ?



  \end{enumerate}



\item Classification. We will use as an example the web-based
  applications Tnasas\footnote{Tnasas is an acronym for ``This is Not A
    Substitute for A Statistician''} (\Burl{http://tnasas.iib.uam.es})
  and GeneSrF   (\Burl{http://genesrf.iib.uam.es}).
  \begin{enumerate}
  \item Look at the class prediction algorithms available. Have we
    mentioned all of them in the class notes? Can you understand what each
    is doing from the help?


  \item Look at the commented example in
    \Burl{http://tnasas.iib.uam.es/Examples/Leukemia.DLDA.F/tnasas-results.html}. Make
    sure you understand it. In particular, understand the meaning of the
    different lines and how we say we can build a predictor. Is the set of
    genes stable?
    
  \item Go now to a four-class example:
    \Burl{http://tnasas.iib.uam.es/Examples/Srbct.SVM.F/results.html}. Notice
    that methods might have changed. Do you understand the output? Is the
    overall error rate helpful to understand how you (miss)classify the
    different groups? Look at the tabular output under ``OOB predictions''
    (almost at the bottom).

    Now go to a different application, but the same data set:
    \Burl{http://genesrf.iib.uam.es/Examples/Srbct/results.html};
    scroll to ``OOB Predictions''; you should see four graphs (if you
    don't, force the browser to refresh the page). If you do not
    understand what each graph is, go to the help (for instance,
    \Burl{http://genesrf.iib.uam.es/help/genesrf-help.html\#f2}). What
    do you see, or don't see, here compared to the output from Tnasas?
    What do you like better and why?

    
  \item Go to an example where no predictor can be built:
    \Burl{http://tnasas.iib.uam.es/Examples/Nothing/results.html}. Do
    you understand what is happening?
    
  \item What would you like to see in this application? What do you think
    is missing? What is misleading?


  \end{enumerate}

\item Prediction with survival data. We will use as an example the
  web-based application SignS{http://signs2.iib.uam.es}.

  \begin{enumerate}
  \item Get familiar with some of the output. Go to the first commented
    example:
    \Burl{http://signs2.iib.uam.es/Examples/CommentedExample/results.html}.  
    Focus on the main ideas (i.e., do not pay too much attention to the
    clustering). What do you think is a bad idea about how the performance
    of the methods is measured?

  \item Go to the second commented example:
    \Burl{http://signs2.iib.uam.es/Examples/InversionSignsCoeffs/results.html}. Make
    sure you understand what is happening.


  \item Among the methods available, which one would you choose, and why?
    No, I am not asking you to go to the literature and look for a review;
    just based on the class notes, which one would you choose? Or which
    one would you not use?
  
  \item How does the purpose and methods of this application differ from
    what you were offered in Pomelo II. For instance, this example output:
    \Burl{http://pomelo2.iib.uam.es/Examples/Cox/results.html}. Check
    the help of Pomelo if you need help:
    \Burl{http://pomelo2.iib.uam.es/help/pomelo2-help.html\#toc9}. 
  
  \item What would you like to see in this application? What do you think
    is missing? What is misleading?
  \end{enumerate}

 
  
\end{enumerate}


% \subsection{Extra activities: only if time permits (and it most likely won't)}
% \label{sec:extra-activ-only}

% As the day of the class approaches, and depending on the calendar, your
% interests, etc, we will discuss which one of these to do. A motivating
% note: the first exercise gives you an option to have your favorite (or
% most disliked) paper examined critically by a group of people.


% \begin{enumerate}


% \item Critical evaluation of the literature that uses methods of
%   differential expression and classification.

%   Identify, preferably in recent papers from high-impact journals (e.g.,
%   Nature, Science, NEJM, Lancet, etc) one paper that uses high-throughput
%   data to build classifiers or identify genes/proteins with differential
%   behavior. You might want to look for papers that are related to your own
%   area of work.  Evaluate the methods used.

%   I do not list the items in your checklist: it is your job to do a
%   thorough evaluation and, thus, to come up with a comprehensive list of
%   items to check. Start using \cite{DupuySimon07} as a guide. I will show
%   you my own checklist when we do the exercise.


% \item Critical evaluation of web-based applications for microarray data
%   analysis.

%   Look at recently modified or created web-based applications for the
%   analysis of microarray data (the web server issue of \textit{Nucleic
%     Acids Research} could help you locate them). Find at least one for
%   differential expression and at least one for classification or survival
%   analysis problems, Evaluate if they are using appropriate approaches.
  
%   I do not list the items in your checklist: it is your job to do a
%   thorough evaluation and, thus, to come up with a comprehensive list of
%   items to check. Start using \cite{DupuySimon07} as a guide. I will show
%   you my own checklist when we do the exercise.


% \item BioConductor and R.

%   Go to the BioConductor page (\Burl{http://www.bioconductor.org}) and
%   locate at least one package that will do analysis of:
%   \begin{enumerate}[(a)]
%   \item Differential expression, with moderated statistics.
%   \item Classification.
%   \item Survival analysis.
%   \item Clustering.
%   \end{enumerate}
  

%   Repeat that exercise going to the R web page
%   (\Burl{http://cran.r-project.org}). You will then need to search for
%   ``packages''. The ``Task views'' are also useful.



% \end{enumerate}




\section{Bibliographic notes}

In addition to the papers you will be reading, you can find many of these
topics discussed at book length in the books by \cite{Dziuda2010} and
\cite{Simon-book-2004} (among many, many others). The book by Dziuda is at
the library. Much more technical books are available (if you know R, I'd
recommend in particular \cite{James2013} which can be freely and legally
downloaded from \Burl{http://www-bcf.usc.edu/~gareth/ISL/}). There are
also a bunch of references in the slides. Another book
filled with statistical common sense and good practice is the excellent
one from F.\ Harrell ``Regresion modeling strategies'', though it is not
directly targeted to high-dimensional problems (but you will learn a lot
about survival analysis from it).


\bibliography{library}


\end{document}


%% very clean run:
%% rm *.idx; rm *.toc; rm *.out; rm *.blg; rm *.log; rm *.aux; rm *.dvi; rm *.bbl; texi2pdf algorithms-class-notes.tex

%%% Local Variables:
%%%   mode: latex
%%%   mode: flyspell
%%%   coding: iso-8859-15
%%% End:



