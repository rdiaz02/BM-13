\documentclass[a4paper,11pt,twoside]{article}
\usepackage[absolute,overlay]{textpos}
\usepackage[english]{babel} %% needed for refs not to get a "~"

\usepackage{lscape}
\usepackage{array}
\usepackage{geometry}
\geometry{verbose,a4paper,tmargin=23mm,bmargin=26mm,lmargin=30mm,rmargin=30mm}

\usepackage{setspace}
\singlespacing

\usepackage{graphicx}

\usepackage[latin1]{inputenc}
\usepackage{times}
\usepackage[T1]{fontenc}

\setcounter{secnumdepth}{5} %% paragraphs in toc
\setcounter{tocdepth}{5}


\usepackage{datetime} %% but I think I am not using it


\usepackage{tikz}
\usetikzlibrary{arrows,shapes,positioning}

\tikzset{
  % Define standard arrow tip
  >=stealth',
  % Define style for boxes
  punkt/.style={
    rectangle,
    rounded corners,
    draw=black, %very thick,
    %maximum width=6em,
    %text width=6.5em,
    minimum height=2em,
    text centered},
  punkt2/.style={
    rectangle,
    rounded corners,
    draw=black, %very thick,
    text width=2.35cm,
    %text width=6.5em,
    minimum height=2em,
    inner sep = 2pt,
    text centered},
  punkt3/.style={
    rectangle,
%    rounded corners,
    draw=black, %very thick,
%    text width=1.cm,
    %text width=6.5em,
    minimum height=2em,
    inner sep = 3pt,
    text centered},
  punkk/.style={
    rectangle,
    rounded corners=3mm,
    draw=black, %very thick,
    %text width=6.5em,
    minimum height=2em,
    text centered},
  ell1/.style={
    %ellipse,
    %draw = black, very thick,
    text width=1.7cm,
    inner sep = 0pt,
    text centered},
  % Define arrow style
  pil/.style={
    ->,
    thick,
    shorten <=2pt,
    shorten >=2pt,}
}




\usepackage{verbatim}
\usepackage{amsmath}
\usepackage[hyphens]{url}
\usepackage[pdftex]{hyperref}
%\usepackage{cite}

\usepackage{url}
\usepackage{xcolor}
\definecolor{light-gray}{gray}{0.72}
\newcommand{\cyan}[1]{{\textcolor {cyan} {#1}}}
\newcommand{\blu}[1]{{\textcolor {blue} {#1}}}
\newcommand{\Burl}[1]{\blu{\url{#1}}}
\newcommand{\Curl}[1]{{\small \url{#1}}} %% you need to scape #

\newcommand{\lgray}[1]{{\textcolor {light-gray} {#1}}}
\newcommand{\red}[1]{{\textcolor {red} {#1}}}
\newcommand{\green}[1]{{\textcolor {green} {#1}}}
\newcommand{\mg}[1]{{\textcolor {magenta} {#1}}}
\newcommand{\og}[1]{{\textcolor {PineGreen} {#1}}}
%\newcommand{\code}[1]{\texttt{\slshape\footnotesize #1}}
%\newcommand{\code}[1]{\texttt{\slshape #1}}
\newcommand{\code}[1]{\texttt{#1}}
\newcommand{\opage}[1]{\texttt{[p.\ #1]}}
\newcommand{\myverb}[1]{{\footnotesize\texttt {\textbf{#1}}}}
\newcommand{\Rnl}{\ +\qquad\ }
\newcommand{\Emph}[1]{\emph{\mg{#1}}}

\providecommand{\tabularnewline}{\\}

%\usepackage{harvard}
\usepackage[authoryear, round, sort]{natbib}
\bibliographystyle{myagsm} %%bbs and chicago does not show URLs 
%%bbs and chicago does not show URLs 

% /texmf/bibtex/bst/myagsm.bst
% and then do texhash

%% Algorithm2e needs to come after natbib!!
\usepackage[boxed,vlined,linesnumbered]{algorithm2e}


%% \usepackage{enumitem} %% not clear I need it
\usepackage{enumerate}

%% an ugly hack, because I cannot get anything else to work

% \newlist{exsection}{enumerate}{10}
% \setlist[exsection]{label*=\thesection.\arabic*.}

% \newlist{exsubsection}{enumerate}{10}
% \setlist[exsubsection]{label*=\thesubsection.\arabic*.}

% \newlist{exsubsubsection}{enumerate}{10}
% \setlist[exsubsubsection]{label*=\thesubsubsection.\arabic*.}


% \DeclareMathOperator*{\argmax}{arg\,max}
% \underset{x}{\operatorname{argmax}}




\title{Systems biology: network reconstruction. \\
\large Class notes for BM-13, ``Bioinform\'atica Avanzada y
Biolog\'ia de Sistemas'', 2014-2015.}


\author{Ram\'on Diaz-Uriarte\\
              Dept. Bioqu\'imica\\Universidad Aut\'onoma de Madrid \\ 
              \texttt{ramon.diaz@iib.uam.es} \\ 
              \Burl{http://ligarto.org/rdiaz} }


%\date{}
% \date{\the\year-\the\month-\the\day}
\date{\gitAuthorDate\ {\footnotesize (Release\gitRels: Rev:
    \gitAbbrevHash)}}



\makeindex

\begin{document}

\maketitle
\vspace{-0.5cm}


\tableofcontents
\clearpage



\section{Introduction}

This is the scenario:


You have read a paper/seen a conference presentation/whatever where
someone showed what they called a ``network''. For instance, what they
called a regulatory network based on gene expression data. You have also
been hearing some of the buzz words, such as ``network motifs'',
``Bayesian networks'', ``ARACNE'', ``partial correlations'', etc.

You are intrigued by this\footnote{You like the pretty pictures too, but
  that should be besides the point} and, since you have several
interesting microarray data sets, you decide to find out what you can get
out of your data if you try this network reconstruction stuff.

Alternatively (and most likely a better reason) you are interested in
understanding how sets of genes relate to each other, since you are
already persuaded that genes do not work one-at-a-time, but sometimes
operate in concert, interact with others, etc.

Regardless what your initial reasons are, you decide to give a try to this
network reconstruction business.  You decide to go to the literature to
find out what these networks are about, and how they can be reconstructed,
and how they can be interpreted. Crucially, you want to know if the
reconstructions are really doing a good job at reconstructing the true
network (i.e., something like whether or not they are recovering ``the
truth''\footnote{I used quotes around ``truth'' because that is/can be a
  slippery concept. A delightful read on this subject is the book ``Truth,
  a guide'', (Oxford University Press, 2005) by the British philosopher
  Simon Blackburn. We also touched upon the ``truth of the model'' issue
  when discussing the ``tell me how it works'' vs.\ the ``black box''
  approaches in classification.}). Thus, you definitely want to read
comparative reviews of methods performance. But, again, once that is taken
care of, you need to worry about interpretation.



\subsection{Pedagogical and scientific objectives of this module}

We only touch here on a very, very minor part of ``systems biology'' but
one that has generated lots of literature (in addition to lots of figures
in papers).  Systems biology is arguably an important research area, but
some people think it might have been oversold, and there is still a huge
``hype factor''.  One specific problem is confounding what we would like
to do (e.g., infer a network of interacting genes) with what we can really
do given our current statistical and computational tools. An important
skill is understanding that getting a picture does not mean: a) that the
stuff in the picture really represents anything real; b) that if there is
anything real in the data, it will be shown in the picture\footnote{This, by the way,
affects other techniques we have seen (e.g., clustering)}.




Some of the major general objectives of this module are to help you
acquire the following habits:

\begin{itemize}
\item Read methodological/statistical reviews of method performance as the
  only reasonable way to decide what  method to use.

\item Systematically ask what it is that the methods are answering. Are
  they giving you what you want? Or something entirely different?

\item Answer the previous question by being willing to go and look in the
  primary literature to really understand what a method is doing.

\item Be critical of the literature you read: there are great papers, and
  there are very poor papers (or poor for your objectives).

\end{itemize}


Some objectives of this module specific to systems biology are:

\begin{itemize}
\item Familiarize you with basic ideas and concepts about networks.

\item Familiarize you with the best methods available for network
  reconstruction.

\end{itemize}



There are no notes for this module as such, only a guide to reading and
asking questions about a few papers.




\section{Activities}


We will be discussing the papers below in class. But the idea is that each
one of you (individually) gives a 10 minute presentation.  In this class
we have gone through:

\begin{itemize}
\item Carefully looking at technical issues and expressions in textbooks
  (DP, HMM, Phylogenies).
\item Taking a quick look at papers from the primary bioinformatics
  literature (Phylogenies, stats).
\item Reading carefully papers from the primary bioinformatics literature
  (stats).
\end{itemize}

After this, you should be ready to do this on your own from beginning to
end. Thus, give a 10 minute presentation where you provide:

\begin{itemize}
\item A brief intro to networks: what they are, types, uses.
\item Discuss/comment on results about method performance. I would focus
  on network reconstruction as in the \cite{Marbach2012},
  \cite{Marbach2010} and \cite{Narendra2011} papers. But you can do
  something else (e.g., boolean networks, signaling, flux balance,
  etc). Just motivate it. We want to know what \textbf{you} would suggest
  the rest of us, based on the literature.
\item If the state of the field sucks, maybe you cannot give any
  recommendation and you are forced to tell us to go do the comparisons
  ourselves, or whatever. Just explain what leads you to this
  recommendation. 
\end{itemize}


% We will be discussing these issues in class, so you should be able to
% provide reasonable answers to the questions. Some of this material might
% be in the exam, too. 


\section{Suggestion for things to think about}

These are suggestions. You should definitely cover these, and these might
give you ideas for your presentation. In addition, we will be discussing
these issues in class, so you should be able to provide reasonable answers
to the questions. Some of this material might be in the exam, too.



\begin{enumerate}
\item Read \textbf{first} the papers that compare network reconstruction
  methods: 
  
  \begin{itemize}\label{net1}

    \item \cite{Marbach2012}
  \item \cite{Marbach2010} %Marbach et al., 2010, \textit{PNAS}
  \item \cite{Narendra2011} %Narendra et al., 2011, \textit{Genomics}
  \end{itemize}

  \begin{enumerate}
    
  \item What are the differences between the data used by both papers? What
    are the advantages and disadvantages of each type of data?
  \item How do the papers differ in their evaluation of quality of
    reconstruction? (these are the ``benchmarks'' or ``performance
    metrics''). Are any of the metrics similar between papers? In what
    situations could each be more interesting?
  \item \cite{Marbach2010} %citeulike:6895598} % Marbach
    talk about recovering ``network motifs'': what does this perspective add
    (or subtracts)? If you do not know what ``network motif'' means, search
    for it (the Wikipedia is a good place to start).
    
  \item ``Simple is good'': is this message common to both papers? Do you
    find anything similar in what we have seen in classification problems
    (previous module)? Could there be any common statistical, biological, or
    computational reason?
  \item If you had to reconstruct a transcriptional network today, what
    method would you use?
    
  \end{enumerate}

\item Read, \textbf{next}, these papers about network reconstruction,
  \textbf{in this order}:
  \begin{itemize}
  \item \cite{Hyduke2010}
  \item \cite{Kim-Transcriptional-2009}
  \item \cite{Butts-networks-2009}
  \end{itemize}

  \begin{enumerate}
  \item Do you notice any change in language between these three papers?
    What about between these three and the previous two? For
    instance, pay attention to expressions such as cis/trans, metrics,
    small-scale, flux-balance. Is this just terminology, or are there
    differences in the concepts being emphasized?
  \item Is there any qualitative change in methodology with signaling networks?
  \item Do you find any comparative \textbf{evidence} of the performance
    of different methods?
  \item What do you get out (if you get anything out) of
    \cite{Butts-networks-2009} when thinking about modeling and
    reconstruction? Explain the sentence ``To represent an empirical
    phenomenon as a network is a theoretical act.'' and its possible
    consequences when comparing the previous two papers (section
    \ref{net1}).
  \item In this set of papers, s the difference between reconstruction and
    analysis clear? What about the previous two? Which field
    (reconstruction vs.\ analysis) do you think is more mature? What are
    the differences and similarities between the formalisms for analysis
    and inference of networks?
  \item Are there any connections with synthetic biology? How could it be
    used to infer networks? Is synthetic biology ever mentioned in the
    previous two papers?
  \end{enumerate}


% \item As the \textbf{one before last} step, read \cite{LeeTzou2009}.
%   \begin{enumerate}
%   \item What does it add?
%   \item Is any information about performance of methods mentioned?
%   \item Does it recommend any method that other papers mention as a poor
%     method?


\item Almost finally \ldots so now you have networks. Different ones, from
  different data. How do you ``integrate'' them? How do you find
  ``modules''?  you care about this, you might want to start by reading
  \cite{Mitra2013}.



\item To really finish now, read \cite{Emmert-Streib2014}. This is about
  the biological interpretation of networks from gene expression data. Do
  they achieve it? What do they propose to do?
% \end{enumerate}


\item \textbf{Finally}, go to the literature and try to locate:
  \begin{enumerate}
  \item Any papers that cite one of the first ones, and that provide
    further, more recent reviews of method performance.
  \item If there are none, then any recent paper that cites one the first
    two, and uses that to guide their choice of methods or their
    development of a new method.
  \end{enumerate}


  For the above activities I usually try PubMed and then Google
  Scholar. Both allow one to search for papers that cite a given
  paper. (From PubMed, if you go to the paper page in the journal where it
  was published, you can often find additional links to get to papers that
  cite that one).


  You should find at least one paper among those two options, read it, and
  be ready to talk about it in class (no, you do not need to
  prepare slides, etc; just be able to share your findings).

\end{enumerate}


\bibliography{library}


\end{document}









%% very clean run:
%% rm *.idx; rm *.toc; rm *.out; rm *.blg; rm *.log; rm *.aux; rm *.dvi; rm *.bbl; texi2pdf algorithms-class-notes.tex

%%% Local Variables:
%%%   mode: latex
%%%   mode: flyspell
%%%   coding: iso-8859-15
%%% End:



